\documentclass[11pt]{article}
\usepackage{setspace}
\linespread{1.25}
\usepackage{lmodern}
\usepackage{amssymb,amsmath}
\usepackage{mathtools}
\usepackage{ifxetex,ifluatex}
\usepackage{fixltx2e} % provides \textsubscript
\ifnum 0\ifxetex 1\fi\ifluatex 1\fi=0 % if pdftex
\usepackage[T1]{fontenc}
\usepackage[utf8]{inputenc}
\else % if luatex or xelatex
\ifxetex
\usepackage{mathspec}
\else
\usepackage{fontspec}
\fi
\defaultfontfeatures{Ligatures=TeX,Scale=MatchLowercase}
\fi
%use upquote if available, for straight quotes in verbatim environments
\IfFileExists{upquote.sty}{\usepackage{upquote}}{}
%use microtype if available
\IfFileExists{microtype.sty}{%
\usepackage{microtype}
\UseMicrotypeSet[protrusion]{basicmath} % disable protrusion for tt fonts
}{}
\usepackage[margin=1in]{geometry}
\usepackage{hyperref}
\hypersetup{unicode=true,
pdftitle={Forecast combination with truncation},
pdfauthor={Daniel Hsiao},
pdfborder={0 0 0},
breaklinks=true}
\urlstyle{same}  % don't use monospace font for urls
\usepackage{graphicx,grffile}
\makeatletter
\def\maxwidth{\ifdim\Gin@nat@width>\linewidth\linewidth\else\Gin@nat@width\fi}
\def\maxheight{\ifdim\Gin@nat@height>\textheight\textheight\else\Gin@nat@height\fi}
\makeatother
%Scale images if necessary, so that they will not overflow the page
%margins by default, and it is still possible to overwrite the defaults
%using explicit options in \includegraphics[width, height, ...]{}
\setkeys{Gin}{width=\maxwidth,height=\maxheight,keepaspectratio}
\IfFileExists{parskip.sty}{%
\usepackage{parskip}
}{% else
\setlength{\parindent}{0pt}
\setlength{\parskip}{6pt plus 2pt minus 1pt}
}
\setlength{\emergencystretch}{3em}  % prevent overfull lines
\providecommand{\tightlist}{%
\setlength{\itemsep}{0pt}\setlength{\parskip}{0pt}}
\setcounter{secnumdepth}{5}
%Redefines (sub)paragraphs to behave more like sections
\ifx\paragraph\undefined\else
\let\oldparagraph\paragraph
\renewcommand{\paragraph}[1]{\oldparagraph{#1}\mbox{}}
\fi
\ifx\subparagraph\undefined\else
\let\oldsubparagraph\subparagraph
\renewcommand{\subparagraph}[1]{\oldsubparagraph{#1}\mbox{}}
\fi

%Use protect on footnotes to avoid problems with footnotes in titles
\let\rmarkdownfootnote\footnote%
\def\footnote{\protect\rmarkdownfootnote}

%Change title format to be more compact
\usepackage{titling}

%Create subtitle command for use in maketitle
\newcommand{\subtitle}[1]{
\posttitle{
\begin{center}\large#1\end{center}
}
}

\setlength{\droptitle}{-2em}
\title{Forecast combination with truncation}
\pretitle{\vspace{\droptitle}\centering\huge}
\posttitle{\par}
\author{Daniel Hsiao}
\preauthor{\centering\large\emph}
\postauthor{\par}
\predate{\centering\large\emph}
\postdate{\par}
\date{November 12, 2018}

\usepackage{placeins}
\usepackage{siunitx}
\usepackage{multirow}
\usepackage{booktabs}


\usepackage{caption}
\captionsetup{font=small}

\begin{document}
\maketitle
\newpage
{
\setcounter{tocdepth}{3}
\tableofcontents
}
\newpage

\section{Introduction}\label{introduction}
Everyone makes predictions. Drawing a line or a curve through the plot, guesses base on intuition, or complex mathematical models are all predictions. However, people don't always make the same forecasts. As a matter of facts, most predictions are different. The more people there are, the more variety the predictions will be. The questions are, "which is better?" and "can we do better than any forecasts alone?"

The study raised from the second question is forecast combinations. The idea is simple: if we don't know which forecast is the best, maybe there are a better solution which is a linear combination of all the forecasts. By combining the forecasts, the prediction contains less idiosyncratic risk, and becomes more the general opinion of all forecasters. This has proven to be successful in decreasing forecast error (\textbf{Clemen 1989, Diebold and Lopez 1996, Chen and Stock and Watson 1999, Dunis and Laws and Chauvin 2000, Stock and Watson 1999}).

The current literature starts with the seminal article of (\textbf{Bates and Granger 1969}) where forecast combination had been introduced to the public. The literature exploded and twenty years later, (\textbf{Clemen 1989}) gathered and reviewed the literature with over 200 items in the bibliography.

In this paper, we look at the correlation between forecasters. The correlation is a strong factor in the weight determination. While many researchers choose to discard correlation due to uncertainty in the covariance estimation, the effect of the correlation can be in favour of the forecast. If both forecaster always over estimates, the high covariance detects this and results in a combined forecast that is not between the two forecasts. Therefore we want to see the effect of negative weights during estimation and to what extend limiting the negative weights can help in the forecast combination.

We propose to limit the negative weights instead of neglecting the correlation at once. We show with survey data from European Central Bank (ECB) that this method can improve the prediction. We impose a threshold parameter, where every negative weights below the threshold are set to zero. Not only does it improve the prediction in terms of mean squared prediction error and mean absolute error, it is also able to outperform equal weight in some area.
 
The result shows that the test statistics\footnote{Lower is better.} decreases in a smooth adjustments and the choice of threshold parameter is not very sensitive to minor changes. This relieves the error during threshold selection, as choosing a slightly wrong threshold does not impact the performance largely. We show that even with in-sample threshold parameter selection, the threshold method still outperforms the equal weights.

We look further in bias correction and simulation study. Bias correction aims to decrease the weights of the forecasts with known bias, while in the simulation study we show the effect of noise-to-signal ratio and correlation on the shape of mean squared prediction error. Bias estimation does not improve in most cases, but improve greatly for one timeseries where threshold does not help. The simulation on the other hand shows that when the correlation increases, the end position of the MSPE decreases. The similar effect can be found in the noise-to-signal ratio. When the noise-to-signal ratio increases, the achievable MSPE with negative weights increases.

In this paper we demonstrated the ability to improve upon equal weight using threshold. The results are promising, but requires further research in areas such as optimal parameter selection.


\section{Methodology}\label{methodology}

\subsection{Standard Approach}\label{standard-approach}

Consider the two variable case here for illustration purpose. We have
two forecasts, \(y_1\) and \(y_2\), of the true variable \(y\). We want
to combine \(y_1\) and \(y_2\) with a weight \(w\) such that we have
\(y_c = w y_1 + (1-w) y_2\). Assume they follow some distribution with finite first and second moment, e.g.
\(y_1 \sim D(0,\sigma_1)\), \(y_2 \sim D(0,\sigma_2)\), and
\(corr(y_1,y_2)=\rho\). Then the variance of the combined forecast
\(y_c\) is

\begin{equation}
\label{eqn: var yc}
Var(y_c) = w^2\sigma_1^2+ (1-w)^2\sigma_2^2+2w(1-w)\sigma_1\sigma_2\rho,
\end{equation}

and the optimal weight with minimal variance is

\begin{equation}
\label{eqn: simple weight}
w^*=\frac{\sigma_2^2-\sigma_1\sigma_2\rho}{\sigma_1^2+\sigma_2^2 -2\sigma_1\sigma_2\rho}.
\end{equation}

Equation \ref{eqn: simple weight} is the standard benchmark approach in
the combination theory, where extensive researches had been done on. 
\textbf{add research}. We refer this weight as the simple weight. Equation \ref{eqn: simple weight} has a few
empirical results that are against this approach. Two common
alternative solutions are diagonal covariance matrix and equal weight.

Ignoring the correlation term \(\rho\) by setting \(\rho=0\), we get the
inverse relation on the variance

\begin{equation}
\label{eqn: simple weight no corr}
w=\frac{\sigma_2^2}{\sigma_1^2+\sigma_2^2}.
\end{equation}

This is a robust way to avoid the estimation of the covariance when the
dimension goes up. The amount of parameters to estimate for the
covariance with dimension \(n\) is \(\frac{1}{2}n(n+1)\), which is
quadratic in \(n\). When the user only estimates the variances, the
amount of parameters to estimate reduces to \(n\), which greatly
decreases the estimation error. \textbf{add citation}

Applying more restriction by setting $\sigma_1=\sigma_2$, we get equal weight. Equal weight is another common approach that works better empirically.
\textbf{add citation} The forecast combination is in this case just an
arithmetic mean of all forecasts. The reason behind using equal weight is the fact
that estimating weight increases or shifts the forecast errors due to
additional estimation error in the estimation of \(w\). This estimation error has a negative effect on the forecast.


\subsection{\texorpdfstring{With estimation error of
\(w\)}{With estimation error of w}}\label{with-estimation-error-of-w}

We can also consider the weight as non-deterministic, but related with
\(y\), e.g., in a trivariate distribution with finite third and fourth
moments. Under trivariate distribution, the variance of the weight
influences the expected value and the variance of the combined forecast.
The expected value and the variance of the combined forecast becomes

\begin{equation}
\label{eqn: E var yc w/ var w}
\begin{aligned}
E(y_c) =& \mu + (cov(w, y_1-y_2))^2\\
var(y_c) =& E(w)^2\sigma_1^2 + (1-E(w))^2\sigma_2^2 + 2E(w)(1-E(w))\rho\sigma_1\sigma_2 \\
+& E[(w-E(w))(y_1-y_2) (E(w)y_1 + (1-E(w))y_2 - \mu)] \\
+& E[(w-E(w))^2 (y_1-y_2)^2] - cov(w,y_1-y_2)^2.
\end{aligned}
\end{equation}

Equation \ref{eqn: E var yc w/ var w} shows the general case of the
forecast combination. If the covariance between \(w\) and \(y_1-y_2\) is
not \(0\), the forecast is biased when combining, with bias
\(cov(w, y_1-y_2)^2\). The variance also increases from equation
\ref{eqn: var yc} with
\(E[(w-E(w))(y_1-y_2) (E(w)y_1 + (1-E(w))y_2 - \mu)]+E[(w-E(w))^2 (y_1-y_2)^2] - cov(w,y_1-y_2)^2\).

This case the only requirement is the individual forecast has to be
unbiased and that the weight sums up to 1.

Let \(d = (d_1, d_2)'\) be the third moment between \(y_1\), \(y_2\),
and
\(\begin{bmatrix} \sigma_{11} & \sigma_{12}\\ \sigma_{12} & \sigma_{22}\end{bmatrix}\)
be the (co)variance matrix, we have

\begin{equation}
\label{eqn: w w/ var w}
\begin{aligned}
w^\dagger = w^*(1+\frac{\sigma_{22} d_1 + \sigma_{11} d_2 -\sigma_{12} (d_1 + d_2)}{\sigma_{11}\sigma_{22} - \sigma_{12}^2}) - \frac{\sigma_{22} d_1 - \sigma{12}d_2}{\sigma_{11}\sigma_{22} - \sigma_{12}^2}.
\end{aligned}
\end{equation}

The non-deterministic weight selection is a linear combination of the
original weight. The non-deterministic weight does not change if the
third moments are \(0\).

\subsection{Negative weight}\label{negative-weight}

Looking back to equation \ref{eqn: simple weight}, we examine the effect
of high correlation term. Assume without loss of generality that
\(\sigma_1 =\sigma_2 (1 + \delta)\), where \(\delta>0\), we rewrite the
weight as

\begin{equation}
\label{eqn: w high corr}
w^* = \frac{(-\rho\delta+ (1-\rho))}{2(1-\rho - (1-\rho)\delta)+\delta^2}.
\end{equation}

The numerator in \(w\) consist of a weighted
mean between $-\delta$ and $1$ with weight \(\rho\) and $1-\rho$ respectively. When \(\rho\)
is small, the weight is close to equation
\ref{eqn: simple weight no corr}, which is $\frac{1}{1+(1+\delta)^2}$. When \(\rho\) is large, the negative in numerator, \(-\delta\), takes over and results in negative
weight. The weight becomes 
\begin{equation}
\label{eqn: w simple rho 1}
w^* = \frac{-\delta}{\delta^2}
\end{equation}
when $\rho$ approaches 1. As $\delta$ approaches 0, the weight quickly goes toward $-\infty$. This is in contrary with the intuition of equal weights, where the variance are equal.
Only when $\delta >1$, that is $\sigma_1 > 2\sigma_2$, will the weight be above -1.
The boundary case is

\begin{equation}
\label{eqn: corr boundary}
\rho = \frac{\sigma_2}{\sigma_1},
\end{equation}

which \(w\) decreases to \(0\) and \(y_c = y_2\).

From equation \ref{eqn: w w/ var w}, we look in the scaling parameter
and the intercept adjustment vector. Under same condition where
\(\sigma_1 =\sigma_2 (1 + \delta)\), we rewrite the determinant to

\begin{equation}
\sigma_{11}\sigma_{22} - \sigma_{12}^2 = \sigma_{22} (1+\delta)^2 (1- \rho^2),
\end{equation}

and the scaling factors becomes

\begin{equation}
\label{eqn: scaling factors}
\begin{aligned}
\frac{\sigma_{22} d_1 - \sigma{12}d_2}{\sigma_{11}\sigma_{22} - 2\sigma_{12}} =& \frac{d_1-\rho(1+\delta) d_2}{(1+\delta)^2 (1- \rho^2)}\\
\frac{\sigma_{22} d_1 + \sigma_{11} d_2 -\sigma_{12} (d_1 + d_2)}{\sigma_{11}\sigma_{22} - 2\sigma_{12}} =& \frac{d_1(1- \rho(1+\delta)) + d_2 ((1+\delta)^2-\rho(1+\delta))} {(1+\delta)^2 (1- \rho^2)}.
\end{aligned}
\end{equation}
When $\rho$ is 0, equation \ref{eqn: scaling factors} simplifies to 
\begin{equation}
\label{eqn: scaling factor rho 0}
\begin{aligned}
\frac{\sigma_{22} d_1 - \sigma{12}d_2}{\sigma_{11}\sigma_{22} - \sigma_{12}^2} =& \frac{d_1}{(1+\delta)^2}\\
\frac{\sigma_{22} d_1 + \sigma_{11} d_2 -\sigma_{12} (d_1 + d_2)}{\sigma_{11}\sigma_{22} - \sigma_{12}^2} =& \frac{d_1} {(1+\delta)^2} + d_2.
\end{aligned}
\end{equation}
The effect on the constant term decreases towards 0 when $|\delta| \to \infty$. However, we see that the scaling term does not decrease to 0, with $d_2$ shifting the weight from optimum.


When $\rho$ reaches 1, the determinant becomes 0, and the limit of weight becomes $\infty$ or $-\infty$ depending on the sign of $d_1$. Due to the fact that the sign of the third moment between $w$ and $y$ is unknown, we cannot infer the sign of the scaling factor. Comparing to equation \ref{eqn: w simple rho 1}, where the effect is limited as long as $\delta$ is big enough, the weight in equation \ref{eqn: w w/ var w} is unrestricted and can take any value. This has a negative effect on the estimation as the increase in $\rho$ increases the estimation uncertainty.

\subsection{Truncated weight}\label{truncated-weight}

To avoid the high correlated forecasts, we use truncation on the
variable. The weight estimation is as follows

\begin{equation}
\label{eqn: w trunc}
\tilde{w} = 
\begin{dcases}
w^* &, \text{ if } c<w^*<1-c \\
0 &, \text{ if } w^*<c \\
1 &, \text{ if } w^*>1-c.
\end{dcases}
\end{equation}


Assume that there is no skewness in the joint distribution, e.g.,
\(w^*\) is an unbiased estimator of the true \(w\). The expected value of
the weight \(\tilde{w}\) is

\begin{equation}
\label{eqn: E w trunc}
\begin{aligned}
E(\tilde{w}) &= E(I_{c<w^*<1-c}w^*) + P(w^*>1-c).
\end{aligned}
\end{equation}

The bias is therefore
\begin{equation}
\label{eqn: bias w trunc}
\begin{aligned}
Bias_c(\tilde{w}) = E(\tilde{w}) - E(w^*) &= -E(I_{w^*<c}w^*)+E(I_{w^*>1-c}(1-w^*))
\end{aligned}
\end{equation}

In the first term, we have \(w^*<c<0\), which cancels out the negative
sign and becomes positive. In the second term we have \(w^*>1-c>1\), which
gives a negative value in \(1-w\). In general case where \(w\) can go
above \(1-c\) or below \(c\) and the skewness of \(w\) is not 0, the
bias is non-zero. The changes in going from $c_1$ to $c_2$ with $c_1<c_2$ is
\begin{equation}
\label{eqn: bias w trunc}
\begin{aligned}
Bias_{c_2}(\tilde{w})- Bias_{c_1}(\tilde{w}) &= -E(I_{c_1<w^*<c_2}w^*)+E(I_{1-c_2<w^*<1-c_1}(1-w^*)).
\end{aligned}
\end{equation}

The variance is

\begin{equation}
\label{eqn: var w trunc}
\begin{aligned}
Var(\tilde{w}) &= Var(I_{c<w^*<1-c}w^*+I_{w^*>1-c})\\
&=Var(I_{c<w^*<1-c}w) + Var(I_{w^*>1-c}) + 2 Cov(I_{c<w^*<1-c}w,I_{w^*>1-c})\\
&=Var(I_{c<w^*<1-c}w) + P(w^*>1-c)(1-P(w^*>1-c)) \\&- 2E(I_{c<w^*<1-c}w^*)P(w^*>1-c)
\end{aligned}
\end{equation}

And the change in variance when $c_1$ becomes $c_2$ is

\begin{equation}
\label{eqn: delta var w trunc 1}
\begin{aligned}
Var_{c_2}(\tilde{w}) - Var_{c_1}(\tilde{w}) &=Var(I_{c_1<w^*<c_2}w+I_{1-c_2<w^*<1-c_1}w^*)\\
&-2Cov(I_{c_2<w^*<1-c_2}w^*,I_{c_1<w^*<c_2}w+I_{1-c_2<w^*<1-c_1}w^*)\\
&+P(1-c_2<w^*<1-c_1)(1-P(1-c_2<w^*<1-c_1))\\
&-2\Big(E(I_{c_2<w^*<1-c_2}w^*)P(w^*>1-c_2)-E(I_{c_1<w^*<1-c_1}w^*)P(w^*>1-c_1)\Big)
\end{aligned}
\end{equation}

The mean squared error (MSE) is then given by 
\begin{equation}
\label{eqn: MSE trunc}
\begin{aligned}
MSE_c &= Bias_c(\tilde{w})^2 + Var(\tilde{w})\\
&=(E(I_{1-c_2<w^*<1-c_1}(1-w^*))-E(I_{c_1<w^*<c_2}w^*))^2 + Var(I_{c<w^*<1-c}w)\\ & + P(w^*>1-c)(1-P(w^*>1-c)) - 2E(I_{c<w^*<1-c}w^*)P(w^*>1-c)
\end{aligned}
\end{equation}

Without an assumption on the distribution of the weight, it is hard to tell if the change in truncation will lead to a higher MSE or a lower MSE. We expect that when $c$ is small enough, e.g. $\tilde{w}\approx w^*$, the increase in bias is small compared to the decrease in variance. When the truncation $c$ gradually increases, the decrease of the variance starts to slow down, and the bias starts to increase more. This is due to the quadratic behaviour of MSE. Thus, somewhere between $w^*$ and $\frac{1}{n}$, there will be an optimal trade-off. This is similar to shrinkage estimator \textbf{cite James Stein}. However, truncation set the truncated weight to 0 instead of $c$. This effectively removes the forecast from the list to be combined. By incorporating this truncation style, the weight can be viewed as a variable selection method in the area of forecast combination, where negative weight is empirically inferior to positive weight. 

\subsection{Bias Correction}\label{bias-correction}

Assume for the prediction error from forecast \(i\),
\(\epsilon_i = y - y_i\), that we can decompose it into predictable term
and unpredictable term:

\begin{equation}
\label{eqn: w bias assumption}
\epsilon_i = b_i + \xi_i. 
\end{equation}
with the expected value
\begin{equation}
\label{eqn: bias estimate}
E(\epsilon_i) = b_i.
\end{equation}

Let $b_i \neq 0$, then the expected error is non-zero
\begin{equation}
E(\epsilon_c) = wb_1+(1-w)b_2
\end{equation}
and the variance
\begin{equation}
Var(\epsilon_c) = w^2\sigma^2_{\xi_1} + (1-w)^2\sigma^2_{\xi_2} + 2w(1-w)\rho_{\xi_1,\xi_2}\sigma_{\xi_1}\sigma_{\xi_2}
\end{equation}
where $\sigma_{\xi_i}$ is the variance of $\xi_i$.

Since the expectation is non-zero, we minimise the mean squared error to calculate the weight.
Write \(b_{ij}=b_i*b_j\), then minimizing the mean squared error gives
\begin{equation}
\label{eqn: w bias}
\begin{aligned}
\hat{w} &= \frac{\sigma_{\xi,22}-\sigma_{\xi,12}+b_{22}-b_{12}}{\sigma_{\xi,11}+\sigma_{\xi,22}-2\sigma_{\xi,12}+b_{11}+b_{22}-2b_{12}}.
\end{aligned}
\end{equation}

We study a few special cases here. When bias is equal, $b_1=b_2$, we have $\hat{w}=w^*$. The bias cancels each other out. 

If $\rho_{\xi_1,\xi_2}=0$ and $b_1=0$, we have
\begin{equation}
\begin{aligned}
\hat{w} &= \frac{\sigma_{\xi,22}+b_{22}}{\sigma_{\xi,11}+\sigma_{\xi,22}+b_{22}}
\end{aligned}
\end{equation}
and converge to 1 if $b_2 \to \infty$.

When $\rho_{\xi_1,\xi_2}=0$ and $b_2=0$, we have
\begin{equation}
\begin{aligned}
\hat{w} &= \frac{\sigma_{\xi,22}}{\sigma_{\xi,11}+\sigma_{\xi,22}+b_{11}}
\end{aligned}
\end{equation}
and converge to 0 if $b_1 \to \infty$. Here we see that the effect of bias are similar to variance. The higher the bias, the lower the weight for the forecast.

The final special case is when $\rho_{\xi_1,\xi_2}=0$. We have
\begin{equation}
\begin{aligned}
\hat{w} &= \frac{\sigma_{\xi,22}+b_{22}-b_{12}}{\sigma_{\xi,11}+\sigma_{\xi,22}+b_{11}+b_{22}-2b_{12}}.
\end{aligned}
\end{equation}
A similar idea approaches here to equation \ref{eqn: w high corr}, with $b_1 = b_2 (1+\gamma)$. However, $b_{22}$ cannot be cancelled out
\begin{equation}
\begin{aligned}
\hat{w} &= \frac{\sigma_{\xi,22}-\gamma b_{22}}{\sigma_{\xi,11}+\sigma_{\xi,22}+b_{22}\gamma^2}.
\end{aligned}
\end{equation}
The implication on the optimal weight is also different from equation \ref{eqn: w high corr}. When $\gamma \to 0$, we have the special case with $b_1=b_2$, which gives the simple weight $w^*$. When $\gamma \to \infty$, the denominator becomes huge and $\hat{W} \to 0$. It is worth noting that the boundary of 0 and $\infty$ does not imply $\hat{w} \in [0,\infty]$. One can think of a case where $b_{22}$ is large enough such that $\sigma_{\xi,ii}$ are negligible and the weight becomes $\frac{-\gamma}{\gamma^2}$. 

\section{Survey of Professional Forecasters
(SPF)}\label{survey-of-professional-forecasters-spf}

To illustrate the empirical results, we use the data from European Central Bank (ECB)
\textbf{(footnote link to data)} in this paper. The data, Survey of Professional Forecasters (SPF), is a
quarterly survey initiated by ECB, with the aim to obtain future
estimates on inflation (HICP), Real Gross Domestic Product year-on-year growth (RGDP) and unemployment rate (UNEM) from
the private sector. Every quarter, a group of professional forecasters
from financial and non-financial institution, such as economic research
institutions, respond to the survey with the idea on the future
economy. Starting 1999, SPF is the longest survey of macroeconomic
expectation in the Euro area. Until the date of this paper, there are 75
quarters of observation available, with 1999 Q4 as the first forecasted
value, and 2018 Q2 as the last observed true macroeconomic indices. We take 2016 Q1 as the first quarter to forecast, with a extending window size. The first in-sample time frame is forecasts made on 1999 Q4 to 2015 Q4, and the second in-sample time frame adds 2016 Q1 to the first.

The set up of the survey consists of multiple magnitudes of questions,
ranging from different horizon to different distribution. The
forecasters are asked to provide their point forecast and the
probability of a certain scenario to happen. The survey enables ECB to
do a quantitative assessment on the consensus of the market, like the
distribution statistics and standard deviations. For this paper, we take
the two most answered periods, which is 1 year ahead and 2 years ahead
as our dataset for all HICP, RGDP, and UNEM.

To compare the forecasts with the actual macroeconomics, we obtain the
true value from ECB database \textbf{(footnote link to data)}. The data
cannot be observed from the economic in 100\% accuracy within the first
time frame and exhibits changes to the initial estimates after
revision. We use the final revision of the macroeconomics where
possible, that is, the revision done on 2018 Q2. The use of a final revision is fine because the
original forecast is not the real target to forecast.

Within the datasets, not all forecasters did a forecast every time
period. To avoid singular outliers, we remove all forecasters with a
total forecasted period of less than 24 quarter (6 years). The removal
approach is in-line with \textbf{(ref paper)}.

Following \textbf{(ref paper)}, we calculate the covariance by looking
at the intersection between each forecaster. Let $T_i$ be the set of periods which the $i$-th forecaster has respond to the survey, $T_i \subseteq \{1,\ldots,T\}$, and let $e_{it}$ be the forecast error of $i$-th forecaster on time $t$, $e_{it} \in T_i$. Then the variance/covariance can be calculated with

\begin{equation}
\label{equation}
\sigma_{i,j} = \frac{1}{|T_i \cap T_j|}\sum_{t\in \{T_i \cap T_j\}} e_{it}*e_{jt}
\end{equation}

When there is no intersection between 2 forecasters, we set the
covariance value to 0. When $i=j$, the intersection of $T_i$ and $T_j$ does not influence the variance calculation, and the variance becomes mean squared error. Additionally, we calculate the correlation by
using the covariance divided by the standard deviation. Standard
deviation is obtained from the square root of the diagonal.

\begin{equation}
\label{eqn: cov2cor}
\rho_{i,j} = \frac{\sigma_{i,j}}{\sigma_{i}\sigma_{j}}
\end{equation}

The cleaned up gives us a preliminary view on the SPF data without the
noises.

\begin{figure}[!h]
\centering
\includegraphics{./Images/SPF.pdf}
\caption{Survey of Professional Forecasters data illustration}\label{fig: SPF data illustration}
\end{figure}

In figure \ref{fig: SPF data illustration} and table
\ref{tab: correlation summary statistics} we show the plots of the
forecasts alongside with the true value in the macroeconomics and the
statistics of the covariance of the forecast error. To avoid too many
lines on the figure by plotting all forecasts, we plot only the minimum,
mean, and maximum from the forecasts. We see that there exist a high
consistency across all forecasts, with two years ahead stronger than one
year. The consistency in the forecast is lower in UNEM than the other
two. Furthermore, many true values lie outside of the forecast range,
with RGDP the worse of all three. More values outside of the forecast
range suggest that restricting positive weight may be a strong
limitation in the forecast combination.

We examine the amount of true value outside of forecast range by
looking at the summary statistics. Let model space be an indicator with 1
when the true value is outside of the forecast range, and 0 when the
true value is inside. Table \ref{tab: modelspace summary statistics}
shows the mean of the indicator. From the three macro topics, RGDP has the highest percentage of time periods to be outside the forecast range, 83\%, followed by HICP
with an average of 48\% of the time periods. UNEM with 38\% of the time periods has the lowest percentage to be outside the forecast
range. Changing from 1 year to 2 years generally does not influence the
mean of the indicator a lot. From the results from figure
\ref{fig: SPF data illustration} and table
\ref{tab: modelspace summary statistics}, we expect to have large effect
using truncation in the forecast of RGDP, while HICP and UNEM do not
have a strong effect. We also expect the two years ahead forecast will
be better than the one year ahead.

\begin{table}[!h]
\centering
\caption{Mean of the model space indicator of the forecast. The indicators are split up into different forecast topics and different forecast horizons. From the three macro topic, RGDP has the least chance within the forecast range, followed by HICP. UNEM has the highest chance to be in the forecast range.}
\label{tab: modelspace summary statistics}
\begin{tabular}{lcccccc}
\hline
&\multicolumn{6}{c}{Model Space Indicator}\\
\cmidrule(lr){2-7}
Macro topic & \multicolumn{2}{c}{HICP} & \multicolumn{2}{c}{RGDP} & \multicolumn{2}{c}{UNEM} \\
\cmidrule(lr){2-3} \cmidrule(lr){4-5}\cmidrule(lr){6-7}
Horizon     & 1 year & 2 years & 1 year & 2 years & 1 year & 2 years \\ 
\hline
Mean        & 0.45        & 0.51         & 0.83        & 0.82        & 0.37         & 0.39       \\
\hline
\end{tabular}
\end{table}

As not all forecasters replied to each survey, we show the amount of forecasters per macroeconomic topic in table \ref{tab: amount of forecasters}. The amount of forecasters after filtering is around 70 for 1 year ahead forecasts, and 60 for 2 year ahead forecasts. We also show the amount of replies per forecaster in a summary statistic in table \ref{tab: forecaster summary statis	tics}. For readability purpose, we round the mean statistics to whole number. The minimum amount of forecast are set to 24, and all forecasters with less than 24 forecasts are discarded. The maximum replies are round 70, which is disctinctivly lower than the full reply of 91 observations. With the mean around 50, we have enough observations to do pair-wise covariance, but multivariate covariance remains skeptical.

\begin{table}[!h]
\centering
\caption{Amount of Forecasters. The amount of the forecaster for 1 year ahead forecast is higher than 2 year ahead.}
\label{tab: amount of forecasters}
\begin{tabular}{lcccccc}
\hline
&\multicolumn{6}{c}{Amount of Forecasters}\\
\cmidrule(lr){2-7}
Macro topic & \multicolumn{2}{c}{HICP} & \multicolumn{2}{c}{RGDP} & \multicolumn{2}{c}{UNEM} \\
\cmidrule(lr){2-3} \cmidrule(lr){4-5}\cmidrule(lr){6-7}
Horizon     & 1 year & 2 years & 1 year & 2 years & 1 year & 2 years \\ 
\hline
Amount        &      72   &60          &70         &64         & 65         & 53       \\
\hline
\end{tabular}
\end{table}

\begin{table}[!h]
	\centering
	\caption{Summary statistics of the survey reply amounts per forecasters. The average forecasters made around 50 replies for the whole duration of the survey. No forecaster replied to all survey. Means are round down to whole number for readability.}
	\label{tab: forecaster summary statistics}
	\begin{tabular}{lcccccc}%{S[table-format=3.2]}
		\hline
		&\multicolumn{6}{c}{Summary Statistics of the Amounts Per Forecaster}\\
		\cmidrule(lr){2-7}
		Macro topic & \multicolumn{2}{c}{HICP} & \multicolumn{2}{c}{RGDP} & \multicolumn{2}{c}{UNEM} \\
		\cmidrule(lr){2-3} \cmidrule(lr){4-5}\cmidrule(lr){6-7}
		Horizon     & 1 year & 2 years & 1 year & 2 years & 1 year & 2 years \\ 
		\hline
Minimum & 24    & 24    & 26    & 24    & 25    & 24    \\
First Q & 40    & 40    & 42    & 35    & 40    & 36    \\
Median  & 55    & 52    & 55    & 51    & 53    & 52    \\
Mean    & 53    & 50    & 54    & 49    & 52    & 50    \\
Third Q & 67    & 62    & 66    & 63    & 66    & 63    \\
Maximum & 74    & 70    & 75    & 71    & 74    & 70      \\ 
		\hline
	\end{tabular}
\end{table}

Table \ref{tab: time summary statistics} shows the summary statistics of the amount of survey replies per time period. On average there are 40 replies per time period. The deviation to the time is small with more than 50\% within 35 to 45 range. The minimum of 22 in UNEM pose a possible problem in the estimation of the covariance. If the estimation of the covariance is bad, limiting negative weights will pose a better result.

\begin{table}[!h]
	\centering
	\caption{Summary statistics of the survey reply amounts per time period. There are around 35 to 45 responses for half of the time. Means are round down to whole number for readability.}
	\label{tab: time summary statistics}
	\begin{tabular}{lcccccc}%{S[table-format=3.2]}
		\hline
		&\multicolumn{6}{c}{Summary Statistics of the Amounts Per Time Period}\\
		\cmidrule(lr){2-7}
		Macro topic & \multicolumn{2}{c}{HICP} & \multicolumn{2}{c}{RGDP} & \multicolumn{2}{c}{UNEM} \\
		\cmidrule(lr){2-3} \cmidrule(lr){4-5}\cmidrule(lr){6-7}
		Horizon     & 1 year & 2 years & 1 year & 2 years & 1 year & 2 years \\ 
		\hline
Minimum & 37    & 26    & 38    & 28    & 32    & 22    \\
First Q & 43    & 35    & 42    & 36    & 38    & 30    \\
Median  & 45    & 38    & 45    & 40    & 40    & 34    \\
Mean    & 46    & 39    & 45    & 40    & 41    & 34    \\
Third Q & 49    & 43    & 48    & 45    & 44    & 38    \\
Maximum & 54    & 49    & 54    & 52    & 51    & 44       \\ 
		\hline
	\end{tabular}
\end{table}

Table \ref{tab: amount per test period} shows the amount of replies per time period that we use to test the model. The values are lower than the mean for almost all time period except a few. Only RGDP 1 year on 2018 Q1 and 2018 Q2, and UNEM 1 year on 2016Q1 are above the average. This decreases the dimension of the covariance matrix, but on the other hand decreases the aomunt of observations. The trade-off between lower dimension should be more important than the loss in observations.

\begin{table}[!h]
	\centering
	\caption{Amount of replies to the survey on the testing period starting 2016. All forecasters that replied in the test periods had also replied in previous time periods.}
	\label{tab: amount per test period}
	\begin{tabular}{lcccccc}%{S[table-format=3.2]}
		\hline
		&\multicolumn{6}{c}{Survey Reply Amounts}\\
		\cmidrule(lr){2-7}
		Macro topic & \multicolumn{2}{c}{HICP} & \multicolumn{2}{c}{RGDP} & \multicolumn{2}{c}{UNEM} \\
		\cmidrule(lr){2-3} \cmidrule(lr){4-5}\cmidrule(lr){6-7}
		Test Period     & 1 year & 2 years & 1 year & 2 years & 1 year & 2 years \\ 
		\hline
2016 Q1      & 45    & 32    & 38    & 33    & 41    & 30    \\
2016 Q2      & 40    & 33    & 42    & 37    & 36    & 28    \\
2016 Q3      & 44    & 35    & 42    & 38    & 39    & 28    \\
2016 Q4      & 45    & 34    & 41    & 40    & 39    & 30    \\
2017 Q1      & 39    & 35    & 38    & 30    & 32    & 32    \\
2017 Q2      & 38    & 27    & 38    & 35    & 33    & 25    \\
2017 Q3      & 37    & 31    & 43    & 36    & 33    & 29    \\
2017 Q4      & 43    & 34    & 41    & 36    & 36    & 29    \\
2018 Q1      & 40    & 27    & 46    & 30    & 34    & 22    \\
2018 Q2      & 44    & 26    & 45    & 31    & 40    & 23    \\ 
		\hline
	\end{tabular}
\end{table}

The table \ref{tab: correlation summary statistics} tell us on how the
forecast errors are correlated. The correlations are split up into
different forecast topics and different forecast horizons. The diagonal
elements of the correlation matrix are not within the table when
generating the summary statistics. For all of the series, the
correlations are on average above 60\%. In HICP and UNEM the correlation
increases across all statistics when the forecast horizon increases,
while RGDP remains the same. The lowest correlation to be found is
-0.02, but this number is not too different from the minimum correlation
in the other two topics.

\begin{table}[!h]
\centering
\caption{Summary statistics of the correlation of the forecast error. The correlation are split up into different forecast topics and different forecast horizons. For all of the series, the correlation are on average above 60\%. In HICP and UNEM the correlation increases across all statistics when the forecast horizon increases, while RGDP remains the same.}
\label{tab: correlation summary statistics}
\begin{tabular}{lcccccc}%{S[table-format=3.2]}
\hline
&\multicolumn{6}{c}{Correlation}\\
\cmidrule(lr){2-7}
Macro topic & \multicolumn{2}{c}{HICP} & \multicolumn{2}{c}{RGDP} & \multicolumn{2}{c}{UNEM} \\
\cmidrule(lr){2-3} \cmidrule(lr){4-5}\cmidrule(lr){6-7}
Horizon     & 1 year & 2 years & 1 year & 2 years & 1 year & 2 years \\ 
\hline
Minimum     & 0.03        & 0.02        & 0.11        & 0.13        & -0.02        & 0.08       \\
First Q     & 0.53        & 0.62        & 0.63        & 0.65        & 0.47         & 0.56       \\
Mean        & 0.64        & 0.70        & 0.75        & 0.75        & 0.59         & 0.67       \\
Median      & 0.66        & 0.72        & 0.81        & 0.80        & 0.62         & 0.70       \\
Third Q     & 0.77        & 0.81        & 0.89        & 0.87        & 0.74         & 0.80       \\
Maximum     & 0.96        & 0.97        & 0.98        & 0.98        & 0.94         & 0.95       \\ 
\hline
\end{tabular}
\end{table}



We conclude in for the analysis of SPF that there exhibits
characteristics that are not in the standard cases where correlation are close to 0. By incorporating the
possibility of negative weight, we expect to see some improvement in the
forecast errors.

\section{Empirical Results}\label{empirical-results}
\subsection{Optimal Weight}\label{optimal-weight}
Analysis of the simple weight given by equation
\ref{eqn: simple weight} provides us with a preliminary understanding of the
variability. Table \ref{tab: simple weight summary statistics} shows the
summary statistics of the optimal weight. We see that for all macroeconomic
topics, the mean and median variate around 0.02, which is close to the
calculation from equal weight. Looking at the first and the third
quantile, we see that half of the weight are between -0.6 and 0.6,
which gives us a possible range of effectiveness of the truncation. The
minimum and the maximum is extreme considering that 1 is 100\%. With
negativity up to -11, -21, and -14, and first and third quartile only up
to -0.6 and 0.6, the weight indicates a strong thick tail behaviour.
Looking further to the difference between 1 year and 2 yearss ahead, we
see that in two out of three macro topics, we see that the extreme
values converge to 0, while RGDP becomes worse. On the other hand, the
gap between first and the third quantile increases for all topics. This
indicates the increase in the choice of negative weight in the general
cases for 2 yearss horizon. The mean and median does not change much in
relation to the forecast horizon.

\begin{table}[!h]
	\centering
	\caption{Summary statistics of the weight from equation \ref{eqn: simple weight}. The mean and median is close to the value from equal weight. However, extreme values exist in the weight, visible in the mimum and maximum. Based on the first and the third quantile, we see that half of the weight are between -0.6 and 0.6.}
	\label{tab: simple weight summary statistics}
	\begin{tabular}{lcccccc}%{S[table-format=3.2]}
		\hline
		&\multicolumn{5}{c}{Optimal Weight}\\
		\cmidrule(lr){2-7}
		Macro topic & \multicolumn{2}{c}{HICP} & \multicolumn{2}{c}{RGDP} & \multicolumn{2}{c}{UNEM} \\
		\cmidrule(lr){2-3} \cmidrule(lr){4-5}\cmidrule(lr){6-7}
		Horizon     & 1 year & 2 years & 1 year & 2 years & 1 year & 2 years \\ 
		\hline
		Minimum      & -11.05      & -6.21      & -9.00      & -21.79      & -14.41      & -5.64      \\
		First Q      & -0.30       & -0.59      & -0.20      & -0.63       & -0.28       & -0.40      \\
		Mean         & 0.04        & 0.04       & 0.01       & -0.01       & 0.03        & 0.05       \\
		Median       & 0.05        & 0.03       & 0.04       & 0.01        & 0.04        & 0.03       \\
		Third Q      & 0.41        & 0.60       & 0.25       & 0.82        & 0.39        & 0.43       \\
		Maximum      & 12.94       & 7.56       & 8.74       & 16.34       & 11.26       & 7.90       \\ 
		\hline
	\end{tabular}
\end{table}
\subsection{Proceure}\label{proceure}

We seek to evaluate the effect of truncating the weight with SPF data.
The procedure to get the weight can be different between different
researcher. Therefore we provide a through order of the steps we take in
the weight estimation. We do the weight estimation for the truncation
using the following steps:

Given each time to forecast, we take all the know observation before
that forecast time period.

\begin{enumerate}
\def\labelenumi{\arabic{enumi}.}
\item
Find the nearest positive definite covariance matrix. This step is
required for the covariance to be invertible. We employ the nearPD
function from r package \emph{Matrix}. The nearPD function first
decompose the covariance into univariate variance and the correlation.
The function then uses the algorithm by \emph{higham} on the
correlation matrix to compute the nearest positive definite matrix.
The final results the the covariance matrix that is combined from the
univariate variance and the correlation matrix.
\item
Subset the covariance. We take the sub matrix containing only the available forecast on the
testing period. That is, if 80 forecasters had made some forecasts
before, but out of them, only 40 has a forecast this time, we discard
the 40 extra forecasters.
\item
Estimate the optimal weight \(w^*\) from equation
\ref{eqn: simple weight}.
\item
Truncate the optimal weight with equation \ref{eqn: w trunc}. The weight now
has less values and does not sum to one. We normalise the weights back such that the sum of the weights is 1.
\item
Combine the forecast using the new weights.
\item
Calculate the test statistics of new weights and equal weights.
\item
Calculate the ratio of the test statistics of new weights over equal weights.
\end{enumerate}

\subsection{Ratio of the Test Statistics}\label{ratio-of-mean-squared-prediction-error}

We evaluate the performance of different weights by mean squared
prediction error (MSPE) and mean absolute prediciton error (MAPE). The truncated MSPE and MAPE are then compared with the
MSPE and from equal weight to produce an test ratio. The results of HICP, RGDP, and UNEM are given in
table \ref{tab: MSPE HICP}, \ref{tab: MSPE RGDP}, and
\ref{tab: MSPE UNEM} respectively. We report the ratio of the MSPE and MAPE. This helps us visualize the effect of truncation instead of comparing per number to the equal weights. A value smaller than 1 means the test statistic of the truncation is smaller than test statistic of equal weight.

We see in table \ref{tab: MSPE HICP} that for the truncation value -2 and -1.5, the minimum of MSPE is attained for 1 year horizon and 2 years horizon respectively. The minimum of MAPE are attended at -2 and 0 respectively. The 1 year selection based on MAPE collides with selection based on MSPE, but the minimum for MAPE is not at -1.5. The optimal weight obtains a ratio of 3.29 and 2.21 for MSPE, significantly larger than the ratio obtained by truncation. The ratio difference of MAPE between optimal weight and truncated weights are smaller compared with MSPE. Truncating all the way to 0 means that no forecast with weight less than 0 are selected. For HICP, selecting too strict threshold does not help in 3 out of 4 cases.

\begin{table}[!h]
	\centering
	\caption{Mean Squared Prediction Error (MSPE) and Mean Absolute Prediction Error (MAPE) of inflation with different truncated value. The MSPE and MAPE of the truncated value are given in the ratio of truncation to equal weight. Larger than 1 means truncated is worse, whereas smaller than 1 means truncated weight helps in reducing MSPE or MAPE.}
	\label{tab: MSPE HICP}
	\begin{tabular}{lcccc}
		\hline
		&                        \multicolumn{4}{c}{HICP}                         \\
		\cmidrule(lr){2-5}                              & \multicolumn{2}{c}{1 Year Horizon} & \multicolumn{2}{c}{2 Years Horizon} \\
		\cmidrule(lr){2-3} \cmidrule(lr){4-5}
		Threshold & MSPE Ratio &    MAPE Ratio    & MSPE Ratio &    MAPE Ratio    \\ \hline
-$\infty$ & 3.2902 & 1.5898 & 2.2072 & 1.4347\\ 
-5 & 2.8211 & 1.4178 & 2.1148 & 1.3537\\ 
-4.5 & 2.8211 & 1.4178 & 2.1148 & 1.3537\\ 
-4 & 2.8187 & 1.414 & 2.1500 & 1.3753\\ 
-3.5 & 2.8178 & 1.4126 & 1.6747 & 1.1988\\ 
-3 & 2.8178 & 1.4126 & 1.5479 & 1.1551\\ 
-2.5 & 2.8180 & 1.4129 & 1.0181 & 1.0638\\ 
-2 & 0.8078 & 0.9025 & 0.993 & 1.0653\\ 
-1.5 & 0.8877 & 0.9619 & 0.9688 & 1.0390\\ 
-1 & 0.8835 & 0.9610 & 0.9764 & 1.0170\\ 
-0.5 & 0.9465 & 0.9912 & 0.9877 & 1.0108\\ 
0 & 0.9844 & 0.9994 & 0.9834 & 1.0005\\ 		 \hline
	\end{tabular}
\end{table}

Table \ref{tab: MSPE RGDP} shows a similar pattern in RGDP as in HICP. The optimal threshold for 1 year is around -1.5, obtaining 0.83 and 0.88 for MSPE and MAPE respectively. We did not look for threshold larger than 0, and the minimum MSPE of RGDP 2 years is not within the search region. This cut-off gives us 1.01 as the boundary case. The MAPE on 2 year horizon acts differently. The minimum MAPE is attended at -4.5. The optimal weight MSPE ratio in 1 year horizon is similar to equal weight, scoring 1.44 in the ratio. 2 years ahead MSPE ratio in other hand does not perform that well in optimal weight, with a ratio of 6.07. The optimal weight in MAPE ratio for 1 and 2 year show similar values to equal weights, scoring 1.07 and 1.80 respectively.

\begin{table}[!h]
	\centering
	\caption{Mean Squared Prediction Error (MSPE) and Mean Absolute Prediction Error (MAPE) of economic growth with different truncated value. The MSPE and MAPE of the truncated value are given in the ratio of truncation to equal weight. Larger than 1 means truncated is worse, whereas smaller than 1 means truncated weight helps in reducing MSPE or MAPE.}
	\label{tab: MSPE RGDP}
	\begin{tabular}{lcccc}
		\hline
		& \multicolumn{4}{c}{RGDP}                                                \\
		\cmidrule(lr){2-5}
		& \multicolumn{2}{c}{1 Year Horizon} & \multicolumn{2}{c}{2 Years Horizon} \\
		\cmidrule(lr){2-3} \cmidrule(lr){4-5}
		Threshold & MSPE Ratio &    MAPE Ratio    & MSPE Ratio &    MAPE Ratio    \\ \hline
		\hline
-$\infty$ & 1.4387 & 1.0681 & 6.0654 & 1.8047\\ 
-5 & 1.1000 & 0.9561 & 1.0684 & 0.9496\\ 
-4.5 & 1.1462 & 0.9861 & 1.0683 & 0.9496\\ 
-4 & 1.1462 & 0.9861 & 1.0885 & 0.9621\\ 
-3.5 & 0.8738 & 0.9029 & 1.0765 & 0.9539\\ 
-3 & 0.8768 & 0.9046 & 1.0824 & 0.9579\\ 
-2.5 & 0.8611 & 0.8967 & 1.102 & 0.9899\\ 
-2 & 0.8402 & 0.8824 & 1.0824 & 0.9567\\ 
-1.5 & 0.8278 & 0.8768 & 1.1712 & 1.0438\\ 
-1 & 0.8610 & 0.9102 & 1.1356 & 1.0617\\ 
-0.5 & 0.8999 & 0.9396 & 1.0198 & 1.0114\\ 
0 & 0.9721 & 0.9844 & 1.0073 & 1.0056\\  \hline
	\end{tabular}
\end{table}


When we look further in UNEM, the results show promising ratios. UNEM achieved the lowest MSPE ratio in 1 year and 2 years at -1.5 and -0.5. Comparing to HICP and RGDP, the MSPE ratio of 0.70 and 0.82 are the lowest of all three. The MSPE ratio of 0.70 is also an 40\% decrease in the MSPE, even when the non-truncated is already at 1.16. MAPE ratio shows that in 1 year ahead, optimal weight already outperforms equal weights with 0.89, but with truncation the MAPE ratio drops to 0.74. The MAPE ratio in 2 year ahead decreases to 0.91. UNEM seems to have an effect from truncation that is not explained by the correlation. UNEM does not have a difference in correlation to other macroeconomics topics that cannot be explained by estimation noise. One possible explanation is the optimal choice of the truncation is more consistent and does not vary over time. We evaluate the consistency of the truncation in the out-of-sample selection area.

\begin{table}[!h] 
\centering
\caption{Mean Squared Prediction Error (MSPE) and Mean Absolute Prediction Error (MAPE) of unemployment with different truncated value. The MSPE and MAPE of the truncated value are given in the ratio of truncation to equal weight. Larger than 1 means truncated is worse, whereas smaller than 1 means truncated weight helps in reducing MSPE or MAPE.}
\label{tab: MSPE UNEM}
\begin{tabular}{lcccc}
\hline
& \multicolumn{4}{c}{UNEM}                                                \\
\cmidrule(lr){2-5}
& \multicolumn{2}{c}{1 Year Horizon} & \multicolumn{2}{c}{2 Years Horizon} \\
\cmidrule(lr){2-3} \cmidrule(lr){4-5}
Threshold & MSPE Ratio &    MAPE Ratio    & MSPE Ratio &    MAPE Ratio    \\ \hline
\hline
-$\infty$ & 1.1639 & 0.8899 & 4.3535 & 1.6517\\ 
-5 & 1.0249 & 0.8438 & 2.3189 & 1.3796\\ 
-4.5 & 0.8926 & 0.7991 & 2.3189 & 1.3796\\ 
-4 & 0.8926 & 0.7991 & 2.2467 & 1.3552\\ 
-3.5 & 0.9447 & 0.8189 & 2.0462 & 1.3042\\ 
-3 & 0.9335 & 0.8137 & 1.9501 & 1.2435\\ 
-2.5 & 0.9097 & 0.8050 & 1.9146 & 1.2427\\ 
-2 & 0.7684 & 0.7858 & 1.8718 & 1.2264\\ 
-1.5 & 0.6988 & 0.7399 & 1.6610 & 1.1328\\ 
-1 & 0.7283 & 0.7554 & 1.4222 & 1.0346\\ 
-0.5 & 0.8232 & 0.8471 & 0.8242 & 0.9066\\ 
0 & 0.9346 & 0.9543 & 0.8769 & 0.9466\\ \hline
\end{tabular}
\end{table}



Figure \ref{fig: Ratio sub} shows the MSPE ratio and MAPE ratio for HICP, RGDP, and UNEM with forecast horizon 1 and 2 years. The figures are plotted with a step size of 0.1. From the figure, HICP shows a sudden decrease in both statistics and minor changes afterwards. RGDP 1 year shows a similar pattern with a larger increase towards the end. For RGDP 2 years, there is a spike in both statistics around 1.3. Given that the value never falls below 1, we suspect that the correlation does not capture the error well enough, and there might be a better measure to counter the forecast error. UNEM 1 year is a bit volatile but exhibits a well-defined U-shape. For UNEM 2 years, the ratios drop in a smooth way to the minimum.

\begin{figure}[!h]
	\centering
	\includegraphics{./Images/Ratio.pdf}
	\caption{MSPE Ratio for different macroeconomic topics}\label{fig: Ratio sub}
\end{figure}

To conclude, we see the possibility to improve the MSPE and MAPE by using the truncation in all cases. In the beginning, there is a drop in the test statistics, but closer to the threshold of 0, test statistics increase. The MSPE and MAPE both have a U-shape characteristic, which attribute to variance and bias. By applying truncation, the drop in variance is larger than the increase in bias, and this contributes to the decrease in the test statistics. It is worth noting that all macroeconomic topics has on average better performance than equal weight, which is considered a benchmark that is hard to beat. Two of those  macroeconomic forecasts, HICP 2 years and RGDP 2 year, have a hard to justify improvement in MSPE or MAPE when estimation error is taken into consideration. Other four obtains from 17\% to 30\% decrease when compared to equal weight. The results also show that the decrease in MSPE is not just one point, but a graduate movement. This implies the robustness in truncation selection. Even if the user does not select the most optimal case, but deviates a small amount next to it, the increase in the test statisitcs is relatively contained.

\subsection{Optimal Truncation Selection}\label{out-of-sample-truncation-selection}

In this section, we show the MSPE and MAPE in weight truncation when
the truncation parameter is in-sample optimally selected. To perform the
selection of the truncation parameter, we follow the following step:

\begin{enumerate}
\def\labelenumi{\arabic{enumi}.}
\item
Estimate the optimal weight in equation \ref{eqn: simple weight}.
\item
Calculate the in-sample mean squared error (MSE) with truncated weight, e.g., the first window is 1999 Q4 to 2015 Q4. The
truncation parameter ranges from -10 to 0 with a step of 0.1. We also look into other starting values, namely -5, -2, and -1.
\item
Select the truncation parameter with the lowest in-sample MSE.
\item
Use the selected truncation parameter to combine the predictions.
\item
Calculate the test statistics of new weights and equal weights.
\item
Calculate the ratio of the test statistics of new weights over equal weights.
\end{enumerate}

If there are
multiple truncation points with the same MSE value, we take the
largest truncation value within the same MSE. The reason to
choose largest value is that the threshold is closer to the next
changing point. Assume for example, a weight with a minimum of -1 is
optimal, and any higher weight results in higher MSE. Then all
truncation values between -10 and -1 gives the same minimal MSE. We
therefore choose -1 to record. The choice does not influence the
forecast and is pure for the analysis later.

\subsection{Optimal Truncation Selection Result}\label{out-of-sample-truncation-result}
We evaluate the Optimal Truncation (OT) forecast ability using the same way as section \ref{ratio-of-mean-squared-prediction-error}. The MSPE ratio and the MAPE ratio are given in table \ref{tab: oos mspe}.

We see in table \ref{tab: oos mspe} that OT improves in all
MSPE cases when we compare with equal weights. In MAPE, OT failed to improve HICP 1 year, but obtained positive results in other 5 topics. The truncation works better in RGDP and UNEM. If we compare the value to the MSPE or MAPE of optimal weights, we see that the truncation improves the forecasts for all topics except UNEM 1 year. RGDP 2 years also shows in MSPE that none of the constant truncations works as good as changing truncation. This means that the optimal truncation selection overweight the selection uncertainty. These results verify the idea that the truncation improves the upon equal weight even if the truncation uncertainty can increase the MSPE. 

Table \ref{tab: oos mspe} also illustrates the effect of changing the search area. Smaller search area gives less uncertainty, but are more restrictive in the solution space. For HICP 1 year and UNEM 2 year, changing the search area does not influence the test statistics enough to be visible under 4 digits of precision. For HICP 2 year and RGDP 1 year, being more restictive increases the test statistics, effectively means worse predictions. For RGDP 2 year and UNEM 1 year, being restrictive decreases the prediction error, and therefore MSPE and MAPE. Since if being restrictive or not relies on prior information on the forecast set, it is hard to determine which is a good choice.

\begin{table}[!h]
\centering
\caption{Mean squared prediction error ratio and mean absolute prediction error ratio when the selection of the truncation is optimal. The minimum search to optimal threshold are -10, -5, -2, and -1.}
\label{tab: oos mspe}
\begin{tabular}{lcccccc}
\hline
&\multicolumn{6}{c}{MSPE with Optimal Truncation}\\
\cmidrule(lr){2-7}
Macro topic & \multicolumn{2}{c}{HICP} & \multicolumn{2}{c}{RGDP} & \multicolumn{2}{c}{UNEM} \\
\cmidrule(lr){2-3} \cmidrule(lr){4-5}\cmidrule(lr){6-7}
Horizon     & 1 year & 2 years & 1 year & 2 years & 1 year & 2 years \\ 
\hline
-10 & 0.9670   & 0.9611   & 0.9275   & 0.9558   & 0.9153   & 0.8752   \\ 
-5  & 0.9670   & 0.9611   & 0.9275   & 0.9558   & 0.9153   & 0.8752   \\ 
-2  & 0.9670   & 0.9719   & 0.9319   & 0.9518   & 0.8982   & 0.8752   \\ 
-1  & 0.9670   & 0.9709   & 0.9319   & 0.9949   & 0.8982   & 0.8752   \\ 
\hline
&\multicolumn{6}{c}{MAPE with Optimal Truncation}\\
\cmidrule(lr){2-7}
Macro topic & \multicolumn{2}{c}{HICP} & \multicolumn{2}{c}{RGDP} & \multicolumn{2}{c}{UNEM} \\
\cmidrule(lr){2-3} \cmidrule(lr){4-5}\cmidrule(lr){6-7}
Horizon     & 1 year & 2 years & 1 year & 2 years & 1 year & 2 years \\ 
\hline
-10 & 1.0015   & 0.9928   & 0.9532   & 0.9577   & 0.9320   & 0.9502   \\
-5  & 1.0015   & 0.9928   & 0.9532   & 0.9577   & 0.9320   & 0.9502   \\
-2  & 1.0015   & 0.9968   & 0.9562   & 0.9533   & 0.9260   & 0.9502   \\
-1  & 1.0015   & 0.9988   & 0.9562   & 0.9524   & 0.9260   & 0.9502   \\
\hline
\end{tabular}
\end{table}

To understand the OT more, table
\ref{tab: truncation summary statistics} looks at the selection of the
truncation. The results show that on average, a truncation around -0.5
is selected, indicating a preference in the negative weight. HICP
2 years and RGDP 2 years select on average lower truncation, -0.8 and
-1.0 respectively. UNEM selects -0.2 on average, highest among all. The
minimum across all macro topics are at most -1, with RGDP 1 year, RGDP 2
year, and UNEM 1 year the lowest selected minimum. The first and the
third quantile show that for the most cases, the truncations are not
extreme, ranging between -0.5 to -0.2 in 1 year and -1 to 0 in 2 years
horizon. We notice an increase in the variation when the horizon
increases. This can be attribute to the uncertainty in the horizon adds
uncertainty in the truncation parameter selection.

\begin{table}[!h]
\centering
\caption{Summary statistics of the truncation selected. The available truncation ranges from -10 to 0 with steps of 0.1. Additionally, we add -$\infty$ to the truncation choice, essentially return the non-truncated weight.}
\label{tab: truncation summary statistics}
\begin{tabular}{lcccccc}%{S[table-format=3.2]}
\hline
&\multicolumn{6}{c}{Selected Optimal Truncation}\\
\cmidrule(lr){2-7}
Macro topic & \multicolumn{2}{c}{HICP} & \multicolumn{2}{c}{RGDP} & \multicolumn{2}{c}{UNEM} \\
\cmidrule(lr){2-3} \cmidrule(lr){4-5}\cmidrule(lr){6-7}
Horizon     & 1 year & 2 years & 1 year & 2 years & 1 year & 2 years \\ 
\hline
Minimum     & -1.0        & -2.1        & -3.3        & \textbf{\emph{-4.6}}        & -3.0        & -1.0        \\
First Q     & -0.6        & -1.0        & -0.5        & -1.1        & -0.4        & -0.3        \\
Mean        & -0.5        & -0.8        & -0.5        & -1.0        & -0.4        & -0.2        \\
Median      & -0.4        & -0.6        & -0.2        & -0.4        & -0.2        & -0.2        \\
Third Q     & -0.3        & -0.3        & -0.1        & -0.2        & -0.2        & 0.0         \\
Maximum     & 0.0         & 0.0         & 0.0         & 0.0         & 0.0         & 0.0         \\ 
\hline
\end{tabular}
\end{table}

Figure \ref{fig: fluctuation} gives a visual inspection on the stability of the selection. For all cases, the fluctuation is limited. Excluding the spikes in some plots, the selection ranges within -1.5 to 0. We suspect that by setting the search range from -1.5 to 0 will improve the OOS truncation selection. 

\begin{figure}[!h]
	\centering
	\includegraphics{./Images/Fluctuation.pdf}
	\caption{Truncation selection for different OOS period}\label{fig: fluctuation}
\end{figure}

\subsection{Bias weighting}\label{bias-weighting}

The bias weighting is an interesting case here. If the model in the bias
estimation successfully estimates the predictable part and the
unpredictable error, this can attirbute to a better weight selection
with additional information gained. Our approach to bias estimation is similar to \textbf{Gibbs \& Vasnev} but deviates in the variable used. We estimate the bias by
\begin{equation}
\label{eqn: bias estimation}
\epsilon_{i,t} = \alpha + \gamma_i y_{i,t} + \eta_{i,t}.
\end{equation}
Which we then produce the estimated future forecast bias by taking the expected value
\begin{equation}
E(\epsilon_{i,t+1}) = \alpha + \gamma_i y_{i,t+1}.
\end{equation}
 \textbf{Gibbs \& Vasnev} uses the true macroeconomic data, while we uses the forecast itself. Essentially they are interchangeable. Rewrite equation \ref{eqn: bias estimation} into
\begin{equation}
\begin{aligned}
\epsilon_{i,t} &= \frac{\alpha}{1+\gamma_i}+\frac{\gamma_i}{1+\gamma_i}y_t + \frac{1}{1+\gamma_i}\eta_{i,t}\\
&= \alpha^* + \gamma_i^* y_t + \eta_{i,t}^*,
\end{aligned} 
\end{equation}
and equation \ref{eqn: bias estimation} becomes the equation given by \textbf{Gibbs \& Vasnev}.

Table \ref{tab: MSPE HICP bias}, \ref{tab: MSPE RGDP bias}, and \ref{tab: MSPE UNEM bias} shows the effect of truncation with bias-corrected weight. The estimation of bias increases the estimation error and therefore is another trade-off between estimation error and the bias. Only two out of six topics have a ratio below 1, considerably worse than without bias correction. The two topics are RGDP 2 year and UNEM 1 year, with minimum of 0.83 and 0.56 respectively. For HICP with bias correction, the truncation performs best with the highest value. This is different from no-bias selection.

\begin{table}[!h]
\centering
\caption{Mean Squared Prediction Error (MSPE) of bias-corrected inflation with different truncated value. The MSPE of the truncated value is given in the ratio of truncation to equal weight. Larger than 1 means truncated is worse, whereas smaller than 1 means truncated weight helps in reducing MSPE.}
\label{tab: MSPE HICP bias}
\begin{tabular}{lcccc}
	\hline
	                                                &                        \multicolumn{4}{c}{HICP}                         \\
	\cmidrule(lr){2-5}                              & \multicolumn{2}{c}{1 Year Horizon} & \multicolumn{2}{c}{2 Years Horizon} \\
	\cmidrule(lr){2-3} \cmidrule(lr){4-5}
Threshold & MSPE Ratio & MAPE Ratio  & MSPE Ratio & MAPE Ratio  \\ \hline
-$\infty$ & 2.7747 & 1.7760 & 1.6452 & 1.3735\\ 
-5 & 2.7747 & 1.7760 & 1.6452 & 1.3735\\ 
-4.5 & 2.7747 & 1.7760 & 1.6452 & 1.3735\\ 
-4 & 2.7747 & 1.7760 & 1.6452 & 1.3735\\ 
-3.5 & 2.7747 & 1.7760 & 1.6452 & 1.3735\\ 
-3 & 2.7747 & 1.7760 & 1.5888 & 1.2976\\ 
-2.5 & 2.7747 & 1.7760 & 1.6059 & 1.3276\\ 
-2 & 2.7747 & 1.7760 & 1.5955 & 1.3064\\ 
-1.5 & 2.7747 & 1.7760 & 1.5887 & 1.2871\\ 
-1 & 2.6528 & 1.6894 & 1.1640 & 1.1141\\ 
-0.5 & 1.7696 & 1.3260 & 1.0464 & 1.0365\\ 
0 & 1.1118 & 1.0576 & 1.0221 & 1.0231\\ \hline
\end{tabular}
\end{table}


RGDP shows similar results in 1 year, where truncation helps, but selecting 0 is the best performing one. On the other hand, RGDP 2 years shows that it can achieve a MSPE ratio of 0.83 with a truncation value of -1. The selection from MAPE ratio yields the same truncation. Compare this to RGDP without bias correction, the bias-corrected performs worse in 1 year horizon, while performing better in 2 years horizon. The best truncation also changes to a different value, from -1.5 for 1 year horizon and 0 for 2 years horizon in no correction to 0 and -1 respectively.

\begin{table}[!h]
\centering
\caption{Mean Squared Prediction Error (MSPE) of bias-corrected economic growth with different truncated value. The MSPE of the truncated value is given in the ratio of truncation to equal weight. Larger than 1 means truncated is worse, whereas smaller than 1 means truncated weight helps in reducing MSPE.}
\label{tab: MSPE RGDP bias}
\begin{tabular}{lcccc}
	\hline
	                                                &                        \multicolumn{4}{c}{RGDP}                         \\
	\cmidrule(lr){2-5}                              & \multicolumn{2}{c}{1 Year Horizon} & \multicolumn{2}{c}{2 Years Horizon} \\
	\cmidrule(lr){2-3} \cmidrule(lr){4-5}
	Threshold & MSPE Ratio & MAPE Ratio  & MSPE Ratio & MAPE Ratio  \\ \hline
-$\infty$ & 3.3002 & 1.5457 & 1.0535 & 0.9458\\ 
-5 & 1.7581 & 1.2626 & 1.1094 & 0.9846\\ 
-4.5 & 1.7915 & 1.2894 & 1.1094 & 0.9846\\ 
-4 & 1.7910 & 1.2888 & 1.1069 & 0.9833\\ 
-3.5 & 1.8086 & 1.298 & 1.1079 & 0.9838\\ 
-3 & 1.8101 & 1.2984 & 1.1076 & 0.9837\\ 
-2.5 & 1.8118 & 1.3004 & 1.0691 & 0.9716\\ 
-2 & 1.6089 & 1.3075 & 0.9950 & 0.9438\\ 
-1.5 & 1.4604 & 1.2530 & 0.8630 & 0.8931\\ 
-1 & 1.2764 & 1.1320 & 0.8275 & 0.8753\\ 
-0.5 & 1.0758 & 1.0464 & 0.9437 & 0.9508\\ 
0 & 1.0436 & 1.0264 & 0.9835 & 0.9892\\ \hline
\end{tabular}
\end{table}


With UNEM 1 year as low as 0.56, UNEM achieved is a large improvement to the equal weight case. However, this effect does not show in the 2 years horizon. The 2 years horizon MSPE ratio increased to higher than equal weight.


\begin{table}[!h]
\centering
\caption{Mean Squared Prediction Error (MSPE) of bias-corrected unemployment with different truncated value. The MSPE of the truncated value is given in the ratio of truncation to equal weight. Larger than 1 means truncated is worse, whereas smaller than 1 means truncated weight helps in reducing MSPE.}
\label{tab: MSPE UNEM bias}
\begin{tabular}{lcccc}
	\hline
	                                                &                        \multicolumn{4}{c}{UNEM}                         \\
	\cmidrule(lr){2-5}                              & \multicolumn{2}{c}{1 Year Horizon} & \multicolumn{2}{c}{2 Years Horizon} \\
	\cmidrule(lr){2-3} \cmidrule(lr){4-5}
	Threshold & MSPE Ratio & MAPE Ratio  & MSPE Ratio & MAPE Ratio  \\ \hline
-$\infty$ & 3.7993 & 1.4829 & 21.2312 & 3.0358\\ 
-5 & 3.7993 & 1.4829 & 2.5172 & 1.5149\\ 
-4.5 & 3.7993 & 1.4829 & 2.5172 & 1.5149\\ 
-4 & 3.7993 & 1.4829 & 2.5376 & 1.5293\\ 
-3.5 & 3.7993 & 1.4829 & 2.5287 & 1.5246\\ 
-3 & 3.3185 & 1.3959 & 2.3569 & 1.4439\\ 
-2.5 & 2.3783 & 1.1662 & 2.3491 & 1.4493\\ 
-2 & 0.9828 & 0.8032 & 2.3203 & 1.4382\\ 
-1.5 & 0.7950 & 0.7275 & 1.6368 & 1.2530\\ 
-1 & 0.5564 & 0.6864 & 1.3372 & 1.1226\\ 
-0.5 & 0.7253 & 0.8155 & 1.1141 & 1.0643\\ 
0 & 0.8736 & 0.9201 & 1.0327 & 1.0176\\ \hline
\end{tabular}
\end{table}

Figure \ref{fig: Ratio bias} shows the MSPE ratio and MAPE ratio over the different threshold in higher granularity. HICP does not attend the minimum with threshold up to 0 for both forecast horizon. The same conclusion also holds for RGDP 1 year and UNEM 2 years. On the other hand, RGDP 2 years has a large effect in truncation with bias-correction occurs, while performs badly without bias-correction. One reason for this is that the problem with RGDP is the highly biasses over time, which cannot be corrected using cross-sectional truncation. UNEM 1 year is another macroeconomic topic that perform better with bias correction. We see from the plot that a large area is well below 0.7, relieving the problem of selecting precisely -0.5.

\begin{figure}[!h]
	\centering
	\includegraphics{./Images/Ratio Bias.pdf}
	\caption{MSPE Ratio for different macroeconomic topics with bias correction}\label{fig: Ratio bias}
\end{figure}


All in all, we observed the possibility to combine bias-correction and truncation. Some macroeconomic performs better with only truncation, while some perform better with the addition of bias-correction. A possible explanation is that the truncation has effect in a different direction than the bias-correction. Truncation relies on the high correlation between different $y_i$, while bias-correction relies on the consistency and predictability of the forecast error in a univariate setting. This difference between two approaches explains the different effect between series. If the predictability of the forecast error is small, such that predicting the forecast increases the MSE, adjusting the bias does not help. Of course, the truncation and bias-correction are not mutually independent. An increase in the correlation increases the similarity of the estimation error. In turn, the estimation errors do not cancel out but magnifies. 



\subsection{Simulation}\label{simulation}
In this section, we show different effect on the underlying model influences the MSPE. For simplicity, we select four true weights and change the data generating model regarding correlation and error variance. We start with the true weight as $w=(-0.5,0.3,0,1.2)'$, and proceed with correlation matrix where all off-diagonals are the same value. Covariance matrix follows by setting all variance to 1. The error term is generated with univariate random normal distribution, while the forecasts are generated by multivariate normal. We do not use biased forecast in this simulation. The simulation runs 10 observations per time for 2000 times for each correlation and error variance combination. Then the simulation takes the average of the MSPE.

In figure \ref{fig: simulation} the different effect of correlation and error variance is shown. We select four different correlation, from 0.5 to 0.8, covering most of the correlation we observed in SPF. For the error variance, we show a range of selection from 0.4 to 0.9. Since the data variance is 1, the error variance quickly gives an idea on the fit of each forecast. 

From figure \ref{fig: simulation}, changes in the correlation and error variance does not influence the position where the MSPE drops or rises. Changes in error variance influence trivially on the level of MSPE. Higher error variance leads to higher MSPE. Changes in correlation show the location where the truncation effect converge to. For low correlation, 0.5 and 0.6, there is a distinct upward movement when truncation goes to 0 for all error variance. This forms the U-shape one expect when the bias increases. For high correlation, the effect of truncation does not increase the bias large enough, making the U-shape observable in only a few cases. Other cases with higher error variance show a continuous decline in MSPE. 

Referring to section \ref{bias-weighting}, the changes in the shape can be accounted for the additional estimation error caused by the bias term. In figure \ref{fig: simulation}, it would be similar to plots of correlation 0.7 where error variance goes from 0.7 to 0.9.

For a U-shape to exist, e.g. equal weight is not the best, the correlation and error variance both plays an important role. Higher correlation only allows lower error when finding the optimal truncation parameter, while lower correlation also allows higher error to exist.

\begin{figure}[!h]
	\centering
	\includegraphics{./Images/Simulation.pdf}
	\caption{MSPE for different correlation and different error variance.}\label{fig: simulation}
\end{figure}

\section{Conclusion and discussion}\label{conclusion}
In this paper, we look into the possibility to use truncation to improve the forecast within forecast combination. Simple weight derived with a deterministic weight assumption often show bad empirical behaviour when compared to equal weight by arithmetic mean. Equal weight has a behaviour that the combined forecast cannot exceed the minimum or the maximum of all forecast. This limitation also holds for all weight selection method where the weight cannot be negative. 

Using truncation, five out of six macroeconomic topics have a positive effect on the mean squared prediction error (MSPE) for European central bank's survey of professional forecasters (SPF). SPF is a quarterly survey sent to the private sector to understand their view on inflation (HICP), economic growth (RGDP), and unemployment (UNEM) in the next one and two year. The survey forecasts have a strong correlation within the forecasters, which we show that the true value often lies outside of the minimum/maximum range. Additionally, the increase in the correlation leads to an increase in the negative weight. Therefore the relaxation of no negative weight can help in this aspect.

By truncation, we set the weight below a certain threshold to 0, which removes the forecasts from the combination list. The results indicate a positive effect up to 30\% decrease in MSPE compared to equal weight. 
This is positive news considering the amount of forecaster is relatively large compared to the sample size. 
With up to 120 forecasters and 60 observations, the weight estimation contains a large amount of estimation noise.
This gives the estimated weight a wide variation, with a minimum of -21.79 and maximum of 16.34. On the other hand, the weights calculated by equal weight method are around 0.01.

The truncation effect holds for out-of-sample (OOS) selection on the truncation parameter. The effect decreases from up to 30\% to up to 12\% decrease in MSPE. Contrary to the truncation where the parameter stays constant, OOS selection allows the parameter to vary over time. This attributes to the increase in performance of RGDP with 2 years forecast horizon. In all six macroeconomic cases, we see an improvement in MSPE when compared to equal weight. For the selected truncation, most of the selected parameter are below -1.5, which is well below the original -21.79. Setting the parameter below -1.5 limits the variation, and therefore decreases the variance and the estimation error.

Looking further into the weight selection, it is possible that the characteristic of SPF is dominated by a consistent bias over a certain forecaster. An example would be a case where the forecaster undervalues the growth and the decline, resulting in a forecast closer to 0 than the true value. Therefore we construct the bias correcting forecast weight and applies the truncation analysis on the corrected weight. The result is mix bag of good and bad news. Four out of six macroeconomic topics are worse than equal weight, while without bias correction, we have only 1 truncation weight that is under-performing. There are still declines in the MSPE compared to the non-truncated ones. On the positive news, bias-correction improves the only macro topic that does not outperform equal weight without bias-correction. The MSPE decreases up to 17\% compared with equal weight. Another positive news is an additional decrease of 13\% in UNEM 1 year ahead. The difference with and without bias correction could be in the direction of the changes of the weight. Truncation tries to extend the forecast upon the range by using negative weight, while bias-correction tries to identify the consistent bias within each forecaster.

We also study the different effect of correlation and estimation uncertainty on the shape of the MSPE. Correlation and estimation uncertainty both play important intertwined roles in the truncation. Higher correlation only allows lower error when finding the optimal truncation parameter, while lower correlation also allows higher error to exist. Since estimation uncertainty and correlation can be known before applying truncation, the selection of best truncation using that information is a possible future research topic.

We list a few other points in this paper that require further research. All analytical derivations are done in two forecasts scenario. They can be extended to multivariate setting. 
RGDP 2 years ahead shows an unexpected shape of MSPE. The reason to this shape is unclear.
Knowing the reason to spikes in the truncation selection is helpful as this can help in limiting variation.
Combination of other techniques like shrinkage on the covariance to improve the estimates.

All in all, we demonstrate the ability to improve upon equal weight by truncation. The requirements for the improvement lies in the correlation and forecast uncertainty. When the conditions are met, for example, the SPF data, equal weight is no longer the prime choice in the forecast combination. 



\bibliography{shortbib}

\end{document}
